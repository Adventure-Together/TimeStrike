\documentclass[../main.tex]{subfiles}

\begin{document}

\section{Sentience}
The Sentience is a massive world boss whose goal is to defeat all teams. If the Sentience defeats all teams in the game, the game is over, and you have lost to the Sentience. The Sentience will act at the end of every round

\subsection{Sentience's Turn}
Every round the Sentience will take an action that can drastically alter the landscape, attack units, and more. 

At the end of each round, draw a card from the Sentience deck. One player will take control of the Sentience and play out the Sentience’s turn based on the instructions of the card.

\textit{Note: Sentience actions cannot target units climbing the Sentience unless specifically stated.  Note: Sentience actions cannot target Lost units.}

\subsubsection{Sentience Control}
Most Sentience cards will instruct players to \textbf{ “Determine control”} of the Sentience for that round. 


If a player has become a Myth (see Myth section), that player controls the Sentience. Otherwise, to determine who controls the card players will first look to see who is on the highest tier of the Sentience. The highest Character’s team will have control over the Sentience, meaning they read and play out the card. If Characters are on equal tiers or there is no Character on the Sentience, control is determined by a D20 roll among the tied players, with the highest roller taking control.


Whenever \textbf{“select new Monster”} is stated on a Sentience card, this means the player selects an unused Beast or Brute card to put into play. The Monster unit is then placed on the board as specified by the Sentience card and the corresponding Monster’s card is placed alongside the other active Monster cards.

\subsubsection{Sentience Movement}

Some Sentience cards will require players to move the Sentience. Unless otherwise stated, the Sentience can be moved anywhere on the board as long as the Sentience fits as described below.

Follow these steps when a Sentience moves to a location where there are other units, walls, uneven terrain, or other obstacles.

\begin{enumerate}
    \item Remove all player and Lost units from where the Sentience will be placed.
    \item If the Sentience is placed on a wall, it is destroyed. Remove it from the board.
    \item If the Sentience is placed on uneven land hexes, remove hexes from under its base until its base is resting on at least 5 even hexes.
    \item If the Sentience is placed on a combination of tiles and land, the Sentience can be placed on both at the same time.
    \item If the Sentience is placed on an overhang or bridge, the Sentience can have a maximum of two hexes hanging off the edge in mid-air. Otherwise destroy the overhang or bridge and place the Sentience on whichever hexes were directly below it.
    \item Once the Sentience is placed, all players must place their removed units adjacent to the Sentience in turn order.
    \item Once all units have been placed adjacent to the Sentience, the player who moved the Sentience must take any Lost units that were removed and place them adjacent to the Sentience.
\end{enumerate}

\subsection{When Climbing the Sentience}
Climbing onto the Sentience is a core part of the game and is extremely important for survival, power, and rewards.

Units on the Sentience cannot be targeted by any ally or enemy units not on the Sentience. This includes any action, skill, or loot cards. Units on the Sentience also cannot target other units that are not on the Sentience.

\textit{Note: Contesting for the first tier is the exception to the above rule. See the Contest Action.}

Units on equal or adjacent tiers of the Sentience are considered engaged.

When attacking the Sentience while climbing, players have height advantage on the Sentience and draw 2 loot cards instead of 1 after successfully attacking.

\subsection{Sentience Mobs}

The Sentience will have uniquely named units that we refer to as Mobs. For example, Fear’s Mobs are called tentacles. Mobs will be brought onto the board throughout the game. These Mobs will have special abilities and can increase the strength of the Sentience, add obstacles, and trigger large-scale damage.

\clearpage
\end{document}